\documentclass{article}

\usepackage{titlesec}
\usepackage{titling}
\usepackage[margin=1in]{geometry}


\titleformat{\section}
{\huge\bfseries}
{}
{0cm}
{}[\titlerule]

\titleformat{\subsection}
{\bfseries}
{}
{0cm}
{}

\titleformat{\subsubsection}[runin]
{\bfseries}
{}
{0cm}
{}[---]

\renewcommand{\maketitle}{
\begin{center}
	{\huge\bfseries
	\theauthor}

\vspace*{1cm}

04-03-1993, Vlissingen \\
Wonend in Ede, Gelderland \\
vitominheere@gmail.com -- vitominheere.com \\

\end{center}
}

\begin{document}

\author{Vito Minheere}


\maketitle

\section{\sc Profiel}
Data Engineer/Backend developer met 4 jaar Python ervaring en 3 jaar GCP ervaring. Geinteresseerd in het ontwerpen en uitvoeren van de simpelste en meest effiente oplossing voor data uitdagingen. \\
Ervaring in data en software engineering zorgt er voor dat ik de getransformeerde data aan kan bieden aan developers en eindgebruikers.

\section{\sc Ervaring}
\subsection{Data Engineer - Qlouder/Cloud Technology Solutions, Utrecht} \hfill {\em 04-2019} \\
Data Engineering consultant in het Google Cloud Platform ecosystem \\
Ontwerp Data Pipelines met \emph{Google Cloud Platform} tools. \\
Ontwikkel moderne web applicaties met \emph{Python} maar ook gewerkt met \emph{Java}, \emph{TypeScript}, \\
\emph{SQL} en \emph{NoSQL} databases.
Microservices  \emph{Docker}, \emph{Kubernetes}, \emph{Terraform} \\

\subsection{Data Engineer - Energyworx, Houten} \hfill {\em 05-2018 04-2019} \\
Inhouse engineer werkend aan data ingestion met \emph{Apache Kafka} and data transformation via \emph{DataFlow}. \\
Beheerde de migratie van de grootste US klant naar de nieuwste versie van het dat analyse framework. \\
Ontwikkel business rules met \emph{Pandas} en \emph{Scikit Learn} \\
Ondersteunen van klanten bij het tranformeren van data via hun custom code in het platform. \\

\subsection{Afstudeer Onderzoek - Globalistics, Valencia} \hfill {\em 09-2017 03-2018} \\
Extraction, transformation and loading (ETL) van data afkomstig uit API's.\\
Gebruik verzamelde data in verschillende voorspel modellen om transport prijzen te voorspellen. \\
Ontwikkel een API met \emph{Django} om de voorspelde resultaten op een website te tonen. \\


\subsection{Backend Developer - Nfnty/Conneqtech, Vlissingen} \hfill {\em 08-2016 09-2017} \\
Gewerkt aan meerdere klant project, zowel front(\emph{AngularJS}) als backend(\emph{PHP})
Oplossingen bedenken en uitvoeren voor issues die in het ticket systeem voorkomen.

\section{Certificaten}
\subsection{Google Cloud Professional Cloud Architect}\hfill {\em 09-2020} \\
\subsection{Google Cloud Professional Data Engineer}\hfill {\em 06-2020} \\

\section{Projecten}
\subsection{Data Extraction - Kweekprocess}
Ontsluit ruwe data uit applicaties en apparaten zodat deze getransformeerd kunnen worden en opgeslagen kunnen worden in databases(\emph{BigQuery}) en \emph{Cloud Storage} \\

\subsection{Accounting Tool - Tourisme}
Gebruik data uit het Datalake om prijzen te generen gebaseerd op voorspelde gegevens. \\
Backend voor de berekeningen gebouwd in \emph{NodeJS} en \emph{Typescript}, frontend is gedaan met \emph{Angular}.

\subsection{Datalake - Tourisme}
Extract en transform data vanuit meerdere bronnen naar een Datalake. Datalake wordt gebruikt om dashboards, spreadsheets en applicaties te voorzien van data. \\
Framework voor opnemen van data gebouwd met \emph{Python} gebruik makende van \emph{BigQuery} en \emph{AppEngine} \\
Exporten van data naar spreadsheets liet de klant toe om met meerdere mensen tegelijk te werken in de cloud.

\subsection{Data Processing - NUTS Voorzieningen}
Refactoring the monolith class for transforming data into smaller, easier unit-testable components \\
Analyzing and transforming time series data from IoT smart meters using Pandas and Numpy \\
Supporting customers with updating and applying business rules to their data sources \\

\subsection{Data Science - Container Transport}
Onderzoek naar opbouw van container transport prijzen. \\
Verrijken van API data zodat deze gebruikt kon worden voor modelleren. \\
Prestaties en accuraatheid vergeleken van verschillende modellen binnen \emph{Scikit-learn} . \\
API gemaakt met \emph{Django} waarmee voorspelde resultaten op een website getoond werden. \\

\clearpage

\section{Opleiding}
\subsection{HBO Software Engineering}\hfill {\em 2013-2018} \\
Minor in Data Science
\subsection{MBO Network Beheerder}\hfill {\em 2009-2013} \\

\section{Technische Skills}
{\sl \textbf{Programming Languages:}} Python, Bash, Dart, Typescript \\
{\sl \textbf{Data Technologies:}} Pandas, BigQuery, SQL \\
{\sl \textbf{Operations Technologies:}} Docker, Terraform, Git \\
{\sl \textbf{Web Backend Technologies:}} Firebase, Django, Flask, REST \\
{\sl \textbf{Web Frontend Technologies:}} Flutter, Angular \\

\section{Nevenactiviteiten}
\subsection{Scheidsrechter - KNKF Sectie Powerliften} \hfill {\em 04-2020 - heden} \\
Berechten op het uitvoeren van de onderdelen tijdens een powerlift wedstrijd.

\subsection{Voorzitter wedstrijd bureau - KNKF Sectie Powerliften} \hfill {\em 02-2018 - heden} \\
Zorgen voor een goede verloop van een powerlift wedstrijd door onder andere het verwerken van beurtbriefjes en scoresheet in het computersysteem. \\
Verwerken van de uitslagen van de wedstrijd zodat deze gebruikt kunnen worden bij de prijsuitreiking.

\end{document}
